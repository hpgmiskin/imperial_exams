\documentclass[10pt,a4paper]{extarticle}
\usepackage[margin=1cm,bmargin=1.6cm]{geometry}

\usepackage{color,minted,multicol,fancyhdr}
\pagestyle{fancy}

\begin{document}

%Document Properties
\title{Exam Reference}
\author{Henry Miskin}

%Header and Footer Setup
\fancyhf{}
\renewcommand{\headrulewidth}{0pt}
\lfoot{hpm14@ic.ac.uk}
\cfoot{\thepage}
\rfoot{\today}

%Colour definitions
\definecolor{airforceblue}{rgb}{0.36, 0.54, 0.66}
\definecolor{darkolivegreen}{rgb}{0.33, 0.42, 0.18}
%\color{airforceblue}

%Minted Configurations
\newminted{cpp}{gobble=1}
\newminted{bash}{gobble=0,linenos,numbersep=0pt}

\begin{multicols}{2}
[\section*{Code}]
\subsection*{Key Packages}

\subsubsection*{$\#$include}
\begin{cppcode}
  #include <cstring>
  #include <cctype>
  #include <iostream>
  #include <fstream>
  using namespace std;  
\end{cppcode}

\subsubsection*{$<$cstring$>$}
\begin{cppcode}
  strcmp(const char* str1, const char* str2);
  strcpy(char* destination, const char* source);
  strcat(char* destination, const char* source);
  strlen(const char* str);
  strtok(char* str, const char* delimiters);
\end{cppcode}

\begin{multicols}{2}
[\subsubsection*{$<$cctype$>$}]
\begin{cppcode}
  bool = isalnum(char);
  bool = isalpha(char);
  bool = isnum(char);

  bool = ispunct(char);
  bool = isspace(char);
  bool = iscntrl(char);
  bool = isupper(char);
  bool = islower(char);

  char = tolower(char);
  char = toupper(char);
\end{cppcode}
\end{multicols}

\subsubsection*{$<$iostream$>$}
\begin{cppcode}
  cin.get(char);
  cin.getline (char* s, streamsize n)
  cin.getline (char* s, streamsize n, char delim)

  cout.width(int);
\end{cppcode}

\subsubsection*{$<$fstream$>$}
\begin{cppcode}
  // File Streams
  fstream.open(const char* filename)
  fstream.close()

  // Input Streams
  ifstream input; // Inherits from cin
  input.putback() // Replaces char & moves pointer

  char ch = input.peek();
  int position = input.tellg()
  input.seekg(int position, input.end)

  bool input.eof()      // Check if end of file
  bool input.fail()     // Check for fail bit
  bool input.is_open()  // Check if file open
  
  // Output Streams
  ofstream output;
  output.put(char)

  position = input.tellp()
  input.seekp(int position, input.end)
\end{cppcode}

\vfill
\columnbreak

\subsection*{Pointers}

\subsubsection*{Arithmetic}
\begin{cppcode}
  int *ip;      // Integer pointer
  int a[10];    // Integer array
  ip = &a[3];   // *ip == a[3]
  ip2 = ip + 1; // *ip2 == a[4]
  *ip2 = 4;     // a[4] = 4
\end{cppcode}

\subsubsection*{Heap Allocation}
\begin{cppcode}
  char* reversed = new char[length];
  delete [] reversed;
\end{cppcode}

\subsubsection*{Function Pointers}
\begin{cppcode}
  int (*sumPointer) (int, int);
  sumPointer = sumFunction;
  sumPointer(int, int) == sumFunction(int, int);
\end{cppcode}

\subsection*{Containers}

\begin{tabular}{ll}
\textbf{Name} \hspace{0.6cm} & \textbf{Properties} \\
list & non continuous storage of data \\
vector & single end sequence containers \\
deque & double end sequence containers \\
\end{tabular}

\subsubsection*{Map}
\begin{cppcode}
  map<int, char> myMap;
  map<int, char>::iterator myIterator;

  int myMap.count(key);
  myIterator = myMap.find(key);
  myIterator == myMap.end();    // Not found
  myMap.erase(myIterator);

  key = myIterator->first;
  value = myIterator->second;
\end{cppcode}

\subsubsection*{String}
\begin{cppcode}
  string myString ("This is my string!");

  myString.append(string str);
  myString.append(string str, int subpos, int sublen);
  myString.append(char* str, int n);
  myString.append(int n, charT c);

  myString.to_str();        // To C string
  myString.push_back(char); // Add char to string
  myString.pop_back();      // Remove last char
  string.clear()            // Clear string contents
\end{cppcode}
\end{multicols}

\pagebreak

\begin{multicols}{2}
[\section*{Code Snippets}]


\subsection*{Split String}
\begin{cppcode}
  string str ("Split this sentence into tokens");
  char * cstr = new char [str.length()+1];
  strcpy (cstr, str.c_str());

  char * p = strtok (cstr," ");
  while (p != 0){
    cout << p << '\n';
    p = strtok(NULL," ");
  }

  delete[] cstr;
  return 0;
\end{cppcode}

\subsection*{Sort Array}
\begin{cppcode}
  char string[100];   // String to sort
  char temp;          // Temporary variable
  int i (0), j (0);
  do {
    j = i;
    do {
      if (compare(string[j],string[i])){
        temp = string[j];
        string[j] = string[i];
        string[i] = temp;
      } 
      j++;
    } while (string[j] != '\0');
    i++;
  } while (string[i] != '\0');
\end{cppcode}

\subsection*{Read From File}
\begin{cppcode}
  ifstream ifs ("test.txt", ifstream::in);
  if (ifs) {
    ifs.seekg (0, ifs.end);
    int length = ifs.tellg();
    ifs.seekg (0, ifs.beg);

    char * buffer = new char [length];
    ifs.read (buffer,length);
    ifs.close();

    cout.write (buffer,length);
    delete[] buffer;
  }
\end{cppcode}

\subsection*{Makefile}
\begin{bashcode}
  FLAGS = -std=c++11 -Wall -g

  file: file.o main.o
    g++ $(FLAGS) file.o main.o -o file #$

  file.o: file.cpp file.h
  main.o: main.cpp

  %.o: %.cpp
    g++ $(FLAGS) -c $<
\end{bashcode}

\vfill
\columnbreak

\subsection*{Containers}

\subsubsection*{Loop Through Map}
\begin{cppcode}
  map<string, int> myMap;

  myMap["one"] = 1;
  myMap["two"] = 2;

  for ( const auto &myPair : myMap ) {
      cout << myPair.first << " - ";
      cout << myPair.second << endl;
  }
\end{cppcode}

\subsection*{Characters \& Integers}

\subsubsection*{Conversion}
\begin{cppcode}
  int i (3);
  char ch;
  ch = (char)(((int)'0')+i);  // 48 + 3 = 51
  cout << ch << endl;         // '0' + 3 = '3'
\end{cppcode}

\subsubsection*{Integer Manipulation}
\begin{cppcode}
  int i (1234);
  cout << i/10;   // 123
  cout << i/100;  // 12
  cout << i%10;   // 234
  cout << i%100;  // 34
\end{cppcode}

\vfill

\begin{center}
\scriptsize %footnotesize
\begin{tabular}{rlrlrlrl}
\textbf{Int} & \textbf{Char} & \textbf{Int} & \textbf{Char} & \textbf{Int} & \textbf{Char} & \textbf{Int} & \textbf{Char} \\
0   & NUL & 32  & SP  & 64  & @ & 96  & ` \\
1   & SOH & 33  & ! & 65  & A & 97  & a \\
2   & STX & 34  & " & 66  & B & 98  & b \\
3   & ETX & 35  & \# & 67  & C & 99  & c \\
4   & EOT & 36  & \$  & 68  & D & 100 & d \\
5   & ENQ & 37  & \%  & 69  & E & 101 & e \\
6   & ACK & 38  & \&  & 70  & F & 102 & f \\
7   & BEL & 39  & ' & 71  & G & 103 & g \\
8   & BS  & 40  & ( & 72  & H & 104 & h \\
9   & HT  & 41  & ) & 73  & I & 105 & i \\
10  & LF  & 42  & * & 74  & J & 106 & j \\
11  & VT  & 43  & + & 75  & K & 107 & k \\
12  & FF  & 44  & , & 76  & L & 108 & l \\
13  & CR  & 45  & - & 77  & M & 109 & m \\
14  & SO  & 46  & . & 78  & N & 110 & n \\
15  & SI  & 47  & / & 79  & O & 111 & o \\
16  & DLE & 48  & 0 & 80  & P & 112 & p \\
17  & DC1 & 49  & 1 & 81  & Q & 113 & q \\
18  & DC2 & 50  & 2 & 82  & R & 114 & r \\
19  & DC3 & 51  & 3 & 83  & S & 115 & s \\
20  & DC4 & 52  & 4 & 84  & T & 116 & t \\
21  & NAK & 53  & 5 & 85  & U & 117 & u \\
22  & SYN & 54  & 6 & 86  & V & 118 & v \\
23  & ETB & 55  & 7 & 87  & W & 119 & w \\
24  & CAN & 56  & 8 & 88  & X & 120 & x \\
25  & EM  & 57  & 9 & 89  & Y & 121 & y \\
26  & SUB & 58  & : & 90  & Z & 122 & z \\
27  & ESC & 59  & ; & 91  & [ & 123 & \{ \\
28  & FS  & 60  & < & 92  & \textbackslash  & 124 & | \\
29  & GS  & 61  & = & 93  & ] & 125 & \} \\
30  & RS  & 62  & > & 94  & \textasciicircum  & 126 & \textasciitilde \\
31  & US  & 63  & ? & 95  & \_ & 127 & DEL
\end{tabular}
\end{center}

\begin{center}
\large{ASCII Table}
\end{center}

\end{multicols}
\end{document}